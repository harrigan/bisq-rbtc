\documentclass{beamer}

\usepackage{amssymb}
\usepackage{bold-extra}
\usepackage{booktabs}
\usepackage{ccicons}
\usepackage{fancybox}
\usepackage{mathabx}
\usepackage{multicol}
\usepackage[normalem]{ulem}
\usepackage[round]{natbib}
\usepackage{pigpen}
\usepackage{siunitx}
\usepackage[absolute,overlay]{textpos}
\usepackage{xparse}
\usepackage{wasysym}
\usepackage{xcolor}
\usepackage{xparse}

\definecolor{myred}{HTML}{E41A1C}
\definecolor{myblue}{HTML}{377EB8}
\definecolor{mygreen}{HTML}{4DAF4A}
\definecolor{mypurple}{HTML}{984EA3}
\definecolor{myorange}{HTML}{FF7F00}
\definecolor{mybrown}{HTML}{A65628}
\definecolor{mypink}{HTML}{F781BF}
\definecolor{myyellow}{HTML}{FFFF33}

\usetheme{metropolis}
\setbeamercolor{alerted text}{fg=red, bg=white}
\setbeamercolor{example text}{fg=green, bg=white}
\setbeamercovered{transparent}

\definecolor{lightgray}{RGB}{153, 153, 153}

\newcommand{\mycitep}[1]{
  {\small\textcolor{lightgray}{\citep{#1}}}%
}

\setbeamercovered{transparent}
\setbeamertemplate{caption}{\raggedright\insertcaption\par}
\setlength{\belowcaptionskip}{-10pt}

\newcommand\blfootnote[1]{
  \begingroup
  \renewcommand\thefootnote{}\footnote{#1}
  \addtocounter{footnote}{-1}
  \endgroup
}

% https://tex.stackexchange.com/a/356901/76669
\let\oldquote\quote
\let\endoldquote\endquote
\RenewDocumentEnvironment{quote}{om}
  {\oldquote}
  {\par\nobreak\smallskip
   \hfill(#2\IfValueT{#1}{~---~#1})\endoldquote 
   \addvspace{\smallskipamount}}

\newcommand{\mylogo}[1]{
  \setlength{\TPHorizModule}{1pt}
  \setlength{\TPVertModule}{1pt}
  \begin{textblock}{1}(250,215)
    #1
  \end{textblock}}


\begin{document}

\title{Bisq Support for rBTC (Video)}

\subtitle{An Entry to the Sovrython}

\author[shortname]{
  \includegraphics[page=1,height=0.5cm]{img/avatars/harrigan}~\texttt{@harrigan}
  \and
  \includegraphics[page=1,height=0.5cm]{img/avatars/tomlloyd92}~\texttt{@tomlloyd92}
  \and
  \includegraphics[page=1,height=0.5cm]{img/avatars/hash-guesser}~\texttt{@hash-guesser}}

\date{}

\frame[plain]{\titlepage}

\frame[plain]{
  \begin{itemize}
    \item<1-> Bisq is a decentralised
      exchange.\footnote{\url{https://bisq.network}}
    \item<2-> In this video we demonstrate support for rBTC using
      our fork of Bisq on a regtest
      network.\footnote{\url{https://github.com/harrigan/bisq}}
    \item<3-> Our pull request to the Bisq project was merged on 13th July.\footnote{\url{https://github.com/bisq-network/bisq/pull/5611}}
  \end{itemize}
}

\frame[plain]{
  \begin{block}{}
    \underline{\textbf{Step 1:}}\\
    Alice (on the left) wants to sell rBTC for BTC. She makes a new
    offer to sell \num{1.00}~rBTC.
  \end{block}
}

\frame[plain]{
  \begin{block}{}
    \underline{\textbf{Step 2:}}\\
    Bob (on the right) wants to buy rBTC using BTC. He takes Alice's
    offer.
  \end{block}
}

\frame[plain]{
  \begin{block}{}
    \underline{\textbf{Step 3:}}\\
    Bisq manages the trade, only releasing the security deposits when
    both sides are happy.
  \end{block}
}

\begin{frame}[plain]
  \centering
  That's it!
\end{frame}

\end{document}
